\documentclass{article}
\usepackage{mathtools}%includes amsmath
\usepackage{amssymb}
\usepackage{amsfonts}
\usepackage{amsthm}
\usepackage{braket}
\usepackage{cancel}
\usepackage[margin=0.9in]{geometry}
\usepackage{bm}
\usepackage{xfrac}
\usepackage{url}

% shorthand
\newcommand{\vsp}{\vspace{0.2cm}}
\newcommand{\no}{\cancel}
\renewcommand{\qed}{\ensuremath{\vsp \\ &\hspace{0.15cm} \blacksquare. \\}}
\newcommand{\schrodinger}{\Schr{\"o}dinger}
\newcommand{\abinitio}{{\it ab initio}}
\newcommand{\ehf}{\ensuremath{E_{\mathrm{HF}}}}
\newcommand{\he}{\ensuremath{\hat{H}_e}}
\newcommand{\inter}{\intertext}

% greek letters
\renewcommand{\a}{\alpha}   % alpha
\renewcommand{\b}{\beta}     % beta
\newcommand{\g}{\gamma}  % gamma
\newcommand{\G}{\Gamma}  
\renewcommand{\d}{\delta}    %delta
\newcommand{\D}{\Delta}     
\newcommand{\e}{\epsilon}  %epsilon
\newcommand{\ev}{\varepsilon}  %varepsilon
\newcommand{\z}{\zeta}       %zeta
\newcommand{\h}{\eta}        %eta
\renewcommand{\th}{\theta}   %theta
\newcommand{\Th}{\Theta}     %Theta
\newcommand{\io}{\iota}      %iota
\renewcommand{\k}{\kappa}    %kappa
\newcommand{\la}{\lambda}    %lambda
\newcommand{\La}{\Lambda}    %Lambda
\newcommand{\m}{\mu}         %mu
\newcommand{\n}{\nu}         %nu
\newcommand{\p}{\rho}        %rho
\newcommand{\si}{\sigma}     %sigma
\newcommand{\siv}{\varsigma} %sigma*
\newcommand{\Si}{\Sigma}     %Sigma
\renewcommand{\t}{\tau}      %tau
\newcommand{\up}{\upsilon}   %upsilon
\newcommand{\f}{\phi}        %phi
\newcommand{\F}{\Phi}        %Phi
\newcommand{\x}{\chi}        %chi
\newcommand{\Y}{\ensuremath{\Psi}}
\newcommand{\y}{\ensuremath{\psi}}
\newcommand{\w}{\ensuremath{\omega}}
\newcommand{\W}{\ensuremath{\Omega}}

\newcommand{\BPh}{\ensuremath{\Bra{\Phi}}}
\newcommand{\KPh}{\ensuremath{\Ket{\Phi}}}

% ornaments
\renewcommand{\eth}{\ensuremath{^\text{th}}}
\newcommand{\rst}{\ensuremath{^\text{st}}}
\newcommand{\ond}{\ensuremath{^\text{nd}}}
\newcommand{\dg}{\ensuremath{^\dagger}}
\newcommand{\bigo}{\ensuremath{\mathcal{O}}}
\newcommand{\tl}{\ensuremath{\tilde}}
\newcommand{\op}[1]{\ensuremath{\hat{#1}}}

%dots
\newcommand{\ld}{\ensuremath{\ldots}}
\newcommand{\cd}{\ensuremath{\cdots}}
\newcommand{\vd}{\ensuremath{\vdots}}
\newcommand{\dd}{\ensuremath{\ddots}}
\newcommand{\hole}{\circ}
\newcommand{\ptcl}{\bullet}
\usepackage{stackengine}
\newcommand{\GNO}[1]{\setstackgap{S}{0.7pt}\ensuremath{\Shortstack{\textbf{.} \textbf{.} \textbf{.}}#1\Shortstack{\textbf{.} \textbf{.} \textbf{.}}}}

\begin{document}
\section*{CIS Equations}
For the Configuration Interaction method including only single excitations (CIS), we construct a Hamiltonian matrix in the basis of singly excited determinants to obtain a set of CIS energies.
%\smallskip
%%%%%%%%%%%%%%%%%%%%%%
%%% 	MATRIX STRUCTURE 	%%%%
%%%%%%%%%%%%%%%%%%%%%%
\subsection*{1. Hamiltonian Matrix Structure}
\vspace{-0.4cm}
\begin{align*}
\inter{Recall that the electronic Hamiltonian in $\Phi$-normal order is:}
\hat{H}_e &= \ehf + f_p^q \tl{a}_q^p + \tfrac{1}{4} \, \bar{g}_{pq}^{rs} \, \tl{a}_{rs}^{pq}
\inter{The structure of the full configuration interaction (FCI) Hamiltonian matrix is:}
\mathbf{H} &= [ \bra{\Phi_P} \he \ket{\Phi_Q}], \Phi_P, \Phi_Q \in \{\Phi, \Phi_i^a, \Phi_{ij}^{ab},  \Phi_{ijk}^{abc}, ...  \}. \\
\inter{In the CI Singles (CIS) approximation, we truncate the determinant basis at single excitations which gives the Hamiltonian matrix the following form:}
\mathbf{H} &= \begin{bmatrix}
\BPh \he \KPh & \BPh \he \ket{\Phi_i^a} \\
\bra{\Phi_i^a} \he \KPh & \bra{\Phi_i^a} \he \ket{\Phi_j^b} \\
\end{bmatrix}
\inter{Clearly $\BPh \he \KPh = $ \ehf, the Hartree-Fock energy of the system.}
\mathbf{H} &= \begin{bmatrix}
\ehf & \bra{\Phi} \he \ket{\Phi_i^a} \\
\bra{\Phi_i^a} \he \ket{\Phi} & \bra{\Phi_i^a} \he \ket{\Phi_j^b} \\
\end{bmatrix}
\intertext{Furthermore, we can apply Brillouin's theorem to show that $\bra{\Phi_i^a} \he \ket{\Phi}  = \bra{\Phi_i^a} \he \ket{\Phi} = 0 \, \, \forall \, \, i,a$. }
   \mathbf{H} 
&= 
   \begin{bmatrix}
      \ \ehf           
   & 
      \mathbf{0} 
   \\
      \mathbf{0} 
   & 
      \bra{\Phi_i^a}
          \he 
       \ket{\Phi_j^b} 
    \\
\end{bmatrix} 
\\
%\inter{So we can see that in our basis $\Phi_P \in \{\Phi, \Phi_i^a \}$, }
%\mathbf{H} \mathbf{c_0} &= \ehf \, \mathbf{c_0}, 
%\inter{where $\mathbf{c_0}$ is the first basis vector, corresponding to \KPh. Thus $\KPh$ is an eigenstate of \he \, with eigenvalue \ehf.}
\intertext{Now let's think about the matrix element $ \bra{\Phi_i^a}  \he \ket{\Phi_j^b}$, which we can represent in the second quantization formalism.}
   \bra{\Phi_i^a} 
      \he 
   \ket{\Phi_j^b}
&= 
   \bra{\F}
      \tl{a}_a^i
      \he
      \tl{a}_j^b
   \ket{\F}
\\
&= \ehf \, \BPh \tl{a}_a^i \tl{a}_j^b \KPh + \BPh \tl{a}_a^i \left[f_p^q \tl{a}_q^p + \tfrac{1}{4} \, \bar{g}_{pq}^{rs} \, \tl{a}_{rs}^{pq} \right] \tl{a}_j^b \KPh \\
\intertext{By Slater's first rule,}
&= \ehf \, \d_{ij}\d_{ab} + \BPh \tl{a}_a^i \left[f_p^q \tl{a}_q^p + \tfrac{1}{4} \, \bar{g}_{pq}^{rs} \, \tl{a}_{rs}^{pq} \right] \tl{a}_j^b \KPh \\
\inter{If we define an eigenvalue-shifted correlation Hamiltonian $\hat{H}_c = f_p^q \tl{a}_q^p + \tfrac{1}{4} \, \bar{g}_{pq}^{rs} \, \tl{a}_{rs}^{pq}$,}
&= \ehf \, \d_{ij}\d_{ab} + \BPh \tl{a}_a^i \hat{H}_c \tl{a}_j^b \KPh
\inter{And the corresponding Hamiltonian matrix becomes}
\mathbf{H} &= \begin{bmatrix}
\,\, \ehf           & \mathbf{0} \\
\mathbf{0} & \bra{\Phi_i^a} \hat{H}_c \ket{\Phi_j^b} + \ehf \, \d_{ij}\d_{ab} \\
\end{bmatrix} \\
		     &= \begin{bmatrix}
\,\, \mathbf{0}           & \mathbf{0} \\
\,\, \mathbf{0} & \bra{\Phi_i^a} \hat{H}_c \ket{\Phi_j^b} \\
\end{bmatrix} + \ehf \, \mathbf{I} \\
&= \mathbf{\tl{H}} + \ehf \, \mathbf{I} \\
\intertext{We have constructed an eigenvalue-shifted $\mathbf{\tl{H}}=\mathbf{H} - \ehf \mathbf{I}$ which corresponds to the operator $\hat{H}_c = \he - \ehf$. It has the same set of eigenstates as $\mathbf{H}$ but eigenvalues shifted down by \ehf. Since the first row and column are all zeros, we can clearly remove the first basis vector, which corresponds to the reference determinant $\F$, from the basis and diagonalize $\mathbf{\tl{H}}$ in the basis $\{\Phi_i^a \}$. Physically, this means that $\F$ is an eigenvector of $\hat{H}_c$ with eigenvalue 0, which means that the CIS correlation energy of the ground state $\F$ is 0. In other words, we will have to include higher excitations to observe correlation effects in the reference determinant.}
\end{align*}


%%%%%%%%%%%%%%%%%%%%%%
%%% 	MATRIX ELEMENTS 	%%%%
%%%%%%%%%%%%%%%%%%%%%%
\newpage
\subsection*{2. Hamiltonian Matrix Elements}
\vspace{-0.4cm}
\begin{align*}
\intertext{We'll write and solve for the matrix elements $\bra{\F_i^a} \hat{H}_c \ket{\F_j^b}$ in the second quantization formalism.}
   \tl{\mathbf{H}}_{ijab} 
&= 
   \bra{\F} 
      \tl{a}_a^i  \, 
      \hat{H}_c \,  
      \tl{a}_j^b 
   \ket{\F} 
\\
&= 
   f_p^q 
   \bra{\F} 
     \tl{a}_a^i\,
     \tl{a}_q^p\,
     \tl{a}_j^b 
   \ket{\F}
+
   \tfrac{1}{4}\,
   \bar{g}_{pq}^{rs} 
   \bra{\F} 
      \tl{a}_a^i\,
      \tl{a}_{rs}^{pq}\,
      \tl{a}_j^b 
   \ket{\F}
\\
\intertext{Applying Wick's theorem for $\F$-normal ordered operators,}
&= 
   f_p^q
   \left(
      \no{
         \bra{\F} 
            \GNO{
               \tl{a}_a^i  \, 
               \tl{a}_q^p \,  
               \tl{a}_j^b 
            }
         \ket{\F}
      }
   + 
      \bra{\F} 
         \GNO{
            \overline{
               \tl{a}_a^i  \, 
               \tl{a}_q^p \,  
               \tl{a}_j^b}}
      \ket{\F} 
   \right)
+ 
   \tfrac{1}{4}\,
   \bar{g}_{pq}^{rs}
   \left( 
      \no{
         \bra{\F} 
            \GNO{
               \tl{a}_a^i \, 
               \tl{a}_{rs}^{pq} \,  
               \tl{a}_j^b 
            }
         \ket{\F}
      }
   + 
      \bra{\F} 
         \GNO{
            \overline{
               \tl{a}_a^i  \, 
               \tl{a}_{rs}^{pq} \,  
               \tl{a}_j^b
            }
          }
      \ket{\F}
   \right)
\\
&= 
   f_p^q
   \left(
      0 + \GNO{\overline{\tl{a}_a^i  \, \tl{a}_q^p \,  \tl{a}_j^b}}
   \right) 
+ 
   \tfrac{1}{4}\,
   \bar{g}_{pq}^{rs}
   \left(
      0+ \GNO{\overline{\tl{a}_a^i  \, \tl{a}_{rs}^{pq} \,  \tl{a}_j^b}}
   \right)
\\
&= 
   f_p^q
   \left[
      \GNO{ \tl{a}^{i \ptcl}_{a \hole} \tl{a}^{p \ptcl \ptcl}_{q \ptcl} \tl{a}^{b \hole}_{j \ptcl \ptcl} } + \GNO{ \tl{a}^{i \ptcl}_{a \circ} \tl{a}^{p \hole}_{q \hole \hole} \tl{a}^{b \hole \hole}_{j \ptcl} }
   \right]
+ 
    \tfrac{1}{4}\,
   \bar{g}_{pq}^{rs}
   \left[
      \hat{P}^{(p/q)}_{(r/s)} (\GNO{\tl{a}_{a \hole}^{i \ptcl} \tl{a}^{p \ptcl \ptcl q \hole}_{r \hole \hole s \ptcl} \,  \tl{a}_{j \ptcl \ptcl}^{b \hole \hole}  } )
   \right]
\\
&= 
   f_p^q
   \left[
      \tl{a}^{i \ptcl p \ptcl \ptcl b \hole}_{a \hole q \ptcl j \ptcl \ptcl} + \tl{a}^{i \ptcl p \hole b \hole \hole}_{a \hole q \hole \hole j \ptcl}
   \right] 
+ 
   \tfrac{1}{4}\,
   \bar{g}_{pq}^{rs}
   \left[
      \hat{P}^{(p/q)}_{(r/s)} ( \tl{a}^{i \ptcl p \ptcl \ptcl q \hole b \hole \hole}_{a \hole r \hole \hole s \ptcl j \ptcl \ptcl}  )
   \right] 
\\
&= 
   f_p^q
   \left[
      \tl{a}^{i \ptcl p \ptcl \ptcl b \hole}_{q \ptcl j \ptcl \ptcl a \hole} + \tl{a}^{i \ptcl p \hole b \hole \hole}_{j \ptcl a \hole q \hole \hole}
   \right] 
+ 
   \tfrac{1}{4}\,
   \bar{g}_{pq}^{rs}
   \left[
      \hat{P}^{(p/q)}_{(r/s)} ( \tl{a}^{i \ptcl p \ptcl \ptcl q \hole b \hole \hole}_{s \ptcl j \ptcl \ptcl  a \hole  r \hole \hole}  )
   \right] 
\\
&= 
   f_p^q
   \left[
      \g_j^i \h_a^p \h_q^b - \g_q^i \g_j^p\h_a^b
   \right] 
+ 
   \tfrac{1}{4}\,
   \bar{g}_{pq}^{rs}
   \left[
      \hat{P}^{(p/q)}_{(r/s)}  [ \, \g^i_s \g^p_j \h^q_a \h^b_r \, ] 
   \right] 
\\
&= 
   f_p^q \, 
   [ 
      \g_j^i
      \h_a^p
      \h_q^b
   -
      \g_q^i
      \g_j^p
      \h_a^b
   ]
+
   \tfrac{1}{4}\,
   \bar{g}_{pq}^{rs}\,
   [
      \g^i_s 
      \g^p_j
      \h^a_q
      \h^b_r 
   - 
      \g^i_s 
      \g^q_j
      \h^a_p
      \h^b_r 
   + 
      \g^i_r 
      \g^q_j
      \h^a_p
      \h^b_s 
    - 
       \g^i_r 
       \g^p_j
       \h^a_q
       \h^b_s
    ]  
\\
&=
   f_a^b 
   \g_j^i 
- 
   f_j^i 
   \h_a^b 
+ 
   \tfrac{1}{4} \, 
   [
      \bar{g}_{ja}^{bi} 
   -  
      \bar{g}_{aj}^{bi}
   + 
      \bar{g}_{aj}^{ib} 
   - 
      \bar{g}_{ja}^{ib}] 
\\
&= 
   f_a^b 
   \g_j^i 
- 
   f_j^i 
   \h_a^b 
+  
   \bar{g}_{ja}^{bi}  
\\
\intertext{Recall that the SCF procedure constructs a diagonal Fock matrix whose diagonal elements are orbital energies:}
   f_a^b 
&= 
   f_a^a \d_a^b 
= 
   e_a \d_a^b
\intertext{Furthermore, if indices $i,j$ loop over occupied and $a,b$ loop over virtual orbitals, then $\g_i^j = \d_i^j$ and $\h_a^b = \d_a^b$. So}
\bra{\F_i^a} 
      \hat{H}_c 
   \ket{\F_j^b}
&= 
   (
      e_a
   -
      e_i)
   \d_a^b 
   \d_j^i 
+  
   \bar{g}_{ja}^{bi}  
\\
\intertext{And these are the standard CIS equations.}
\end{align*}

\end{document}