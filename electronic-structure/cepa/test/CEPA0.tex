\documentclass{article}[11pt]
\usepackage[margin=0.9in]{geometry}
\usepackage{mathtools}
\usepackage{cancel}
\usepackage{bm}
%\usepackage{xfrac}


\begin{document}

\section*{Coupled-Electron Pair Approximations (CEPAs)}
The CEPA methods are techniques for computing dynamic correlation energy on top of a Hartree-Fock (HF) reference. For convenience, define a correlation energy $E_{{c}}$ and a correlated Hamiltonian $\hat{H}_{\mathrm{c}}$:
\begin{align*}
E_{{c}} &= E_{\mathrm{e}} - E_{\mathrm{HF}}   &&  \hat{H}_{\mathrm{c}} = \hat{H}_{\mathrm{e}} - E_{\mathrm{HF}} \\
\intertext{Obeying the convention that indices $i,j,k...$ correspond to occupied orbitals and $a,b,c,...$ to virtual orbitals, we also define a doubles excitation operator:}
\hat{T}_2 | \Phi \rangle &= \frac{1}{4}\sum\limits_{ijab} t_{ij}^{ab} | \Phi_{ij}^{ab} \rangle.
\end{align*}
With these definitions and a Hartree-Fock reference $| \Phi \rangle$, the CCD equations are:
\begin{align}
   E_c 
&= 
	\langle \Phi |
      \hat{H}_c
      \hat{T_2}
   | \Phi \rangle
\\
   E_c
   t_{ij}^{ab}
&=
     \langle
   	  \Phi_{ij}^{ab} |
      \hat{H}_c 
   | \Phi \rangle
+
   \langle \Phi_{ij}^{ab} |
      \hat{H}_c
      \hat{T_2}
   | \Phi \rangle
+ 
   \frac{1}{2}
   \langle \Phi_{ij}^{ab}
     | \hat{H}_c
      \hat{T_2}^2 |
    \Phi \rangle \
\end{align}
where (1) is called the CCD energy expression and (2) is the CCD amplitude equations for the excitation $ij\rightarrow ab$. Both expressions can be simplified using Slater rules or Wick's theorem. 
%but for our purposes we'll simplify the $\bra{\F}\hat{H}_{{c}}\hat{T}_2| \Phi \rangle$ and $\frac{1}{2}\langle \Phi_{ij}^{ab} |\hat{H}_{\mathrm{c}}\hat{T}_2^2| \Phi \rangle$ terms:
\begin{align*}
\intertext{The simplified energy expression (1) is:}
   E_{{c}}
&= 
   \cancel{
   \langle \Phi |
      \hat{H}_c
   | \Phi \rangle
   }
+ 
   \frac{1}{4}
   \sum\limits_{klcd}
   t_{kl}^{cd}
   \langle \Phi |
      \hat{H}_c
   | \Phi_{kl}^{cd} \rangle
\\
   E_{{c}}
&= 
   \frac{1}{4}
   \sum\limits_{klcd}
   \langle 
   		kl || cd 
	\rangle
   t_{kl}^{cd}
\intertext{For the amplitude equations, we can split up the $\hat{T}_2^2$ term into linked and unlinked contributions.}
   \frac{1}{2}
   \langle \Phi_{ij}^{ab} |
      \hat{H}_c
      \hat{T}_2^2
   | \Phi \rangle \
&=
   \frac{1}{2}
   \left(
      \frac{1}{4}
   \right)^2
   \sum\limits_{klcd}
   \sum\limits_{mnef}
   t_{kl}^{cd}
   t_{mn}^{ef}
   \langle \Phi_{ij}^{ab}
    | \hat{H}_c |
   \rangle \Phi_{klmn}^{cdef}
\\
&= 
   \Delta_{{U}} 
+ 
   \Delta_{{L}} 
\intertext{where the unlinked ($\Delta_{{U}} $) and linked ($\Delta_{\mathrm{L}}$) contributions are defined as follows: (the proof uses diagrams)}
\\
  \Delta_{{U}} 
&= 
   \frac{1}{4}
   \sum\limits_{klcd}
  \langle kl || cd \rangle
   t_{kl}^{cd}
   t_{ij}^{ab}
\\
   \Delta_{{L}}
&= 
- 
   \frac{1}{2}
   \hat{P}(ab)
   \sum\limits_{klcd}
  \langle kl || cd \rangle
   t_{ij}^{ac}
   t_{kl}^{bd}
- 
   \frac{1}{2}
   \sum\limits_{klcd}
   \hat{P}(ij)
   \langle kl || cd \rangle
   t_{ik}^{ab}
   t_{jl}^{cd}
\\
&+ 
   \frac{1}{4}
   \sum\limits_{klcd}
   \langle kl || cd \rangle
   t_{ij}^{cd}
   t_{kl}^{ab}
+ 
  \hat{P}(ij)
  \sum\limits_{klcd}
  \langle kl || cd \rangle
  t_{ik}^{ac}
  t_{jl}^{bd}
\intertext{Note that $\Delta_{{U}} = E_c t_{ij}^{ab}$, the left-hand side of equation (2). Thus the amplitude equation (2) simplifies to}
   \cancel{
   E_c
   t_{ij}^{ab}
   }
&=
	\langle
   		\Phi_{ij}^{ab}
      	| \hat{H}_c  |
   		\Phi 
   \rangle
+
   \langle
   		\Phi_{ij}^{ab}
      	| 	\hat{H}_c
      		\hat{T_2}
		| 
	\Phi \rangle
+
   \cancel{
    \Delta_{{U}}
    }
+ 
   \Delta_{{L}}
\end{align*}
Applying these simplifications, the energy and amplitude equations can be written as
\begin{align}
    E_c
 &= 
    \frac{1}{4}
   \sum\limits_{klcd}
   \langle kl || cd \rangle
   t_{kl}^{cd}
 \\
    0 
&= 
   \langle \Phi_{ij}^{ab} |
      \hat{H}_c 
   | \Phi \rangle
+
   \langle \Phi_{ij}^{ab} |
      \hat{H}_c
      \hat{T_2}
   | \Phi \rangle
+ 
   \Delta_{{L}}
\end{align}
The CCD expression for $\Delta_{{L}}$ is nonlinear in the $t$-amplitudes (they are multiplied by each other). This is what makes the coupled-cluster equations very complex and difficult to parallelize in a production code. Thus, the goal of the CEPA methods is to include a linear approximation for $\Delta_{\mathrm{L}}$. 
 \newpage

\subsection*{CEPA(0)}
The simplest CEPA technique, known as CEPA(0), simply assumes $\Delta_{{L}} = 0$, yielding the equations
\begin{equation}
    E_c
= 
    \frac{1}{4}
   \sum\limits_{klcd}
   \langle kl || cd \rangle
   t_{kl}^{cd}
\end{equation}
\begin{equation}
    0 
= 
   \langle \Phi_{ij}^{ab} | 
      \hat{H}_c 
   | \Phi \rangle
+
   \langle \Phi_{ij}^{ab} |
      \hat{H}_c
      \hat{T_2}
   | \Phi \rangle
\end{equation}
\\
Note that the energy expression has not changed: (5) = (3) = (1), but the amplitude equations (6) have been simplified to include only linear terms. Although this seems like an oversimplification, the motivation is that the linked contributions to the $\hat{T}_2^2$ matrix element essentially cancel each other out. This has proven to be a fair assumption; CEPA$_0$ often outperforms CCSD in the equilibrium region (albeit due to error cancellation).
%One can apply the Slater Rules (or Wick's theorem) to the first and third terms on the right-hand side of (2):
%\begin{equation*}
%   \langle \Phi_{ij}^{ab} | 
%      \hat{H}_c 
%   | \Phi \rangle
%=
%   \braket{ij || ab} 
%\end{equation*}
%\begin{equation*}
% \sum\limits_{\substack{k<l\\c<d}}
%   \sum\limits_{\substack{m<n\\e<f}}
%   t_{kl}^{cd}
%   t_{mn}^{ef}
%   \langle \Phi_{ij}^{ab} |
%      \hat{H}_c
%   \ket{\Phi_{klmn}^{cdef}}
%= 
%   \sum\limits_{\substack{k<l\\c<d}}^{\neq ijab}
%      t_{kl}^{cd}
%      t_{ij}^{ab}
%      \bra{\F}
%         \hat{H}_c
%      \ket{\Phi_{kl}^{cd}}
%\end{equation*}
%We now have simplified the amplitude equations,
%\begin{equation}
%E_c
%   t_{ij}^{ab}
%=
%   \braket{ij || ab} 
%+
%   \sum\limits_{\substack{k<l\\c<d}}
%   t_{kl}^{cd}
%   \langle \Phi_{ij}^{ab} |
%      \hat{H}_c
%   \ket{\Phi_{kl}^{cd}}
%+ 
%   \frac{1}{2}
%   \sum\limits_{\substack{k<l\\c<d}}
%   \sum\limits_{\substack{m<n\\e<f}}
%   t_{kl}^{cd}
%   t_{mn}^{ef}
%   \langle \Phi_{ij}^{ab} |
%      \hat{H}_c
%   \ket{\Phi_{klmn}^{cdef}} \
%\end{equation}
%\begin{align*}
%   \braket{ij || ab} 
%&+
%   \sum\limits_{\substack{k<l\\c<d}}
%   t_{kl}^{cd}
%   \bra{\F}
%      \hat{H}_c
%   \ket{\Phi_{kl}^{cd}}
%+ 
%   \frac{1}{2}
%   \left[ 
%      \sum\limits_{\substack{k<l\\c<d}}^{\neq ijab}
%      t_{kl}^{cd}
%      t_{ij}^{ab}
%      \bra{\F}
%         \hat{H}_c
%      \ket{\Phi_{kl}^{cd}} \
%   + 
%      \sum\limits_{\substack{m<n\\e<f}}^{\neq ijab}
%      t_{mn}^{ef}
%      t_{ij}^{ab}
%      \bra{\F}
%         \hat{H}_c
%      \ket{\Phi_{mn}^{ef}} \
%   \right ] 
%\intertext{Since summed-over indices are effectively just dummy variables,}
%&= 
%   \braket{ij || ab} 
%+
%   \sum\limits_{\substack{k<l\\c<d}}
%   t_{kl}^{cd}
%   \bra{\F}
%      \hat{H}_c
%   \ket{\Phi_{kl}^{cd}}
%+ 
%   \frac{1}{2}
%   \left[ 
%      2 
%      \times
%      \sum\limits_{\substack{k<l\\c<d}}^{\neq ijab}
%      t_{kl}^{cd}
%      t_{ij}^{ab}
%      \bra{\F}
%         \hat{H}_c
%      \ket{\Phi_{kl}^{cd}} \
%    \right] 
%\end{align*}
%On the lefthand side, substitute the expression for $E_c$ from (1):
%\begin{align*}
%  t_{ij}^{ab} 
%  \left(
%     \sum\limits_{\substack{k<l\\c<d}}
%      t_{kl}^{cd}
%      \bra{\F}
%         \hat{H}_c
%      \ket{\Phi_{kl}^{cd}} \
%   \right)
%&= 
%   \braket{ij || ab} 
%+
%   \sum\limits_{\substack{k<l\\c<d}}
%   t_{kl}^{cd}
%   \bra{\F}
%      \hat{H}_c
%   \ket{\Phi_{kl}^{cd}}
%+ 
%      t_{ij}^{ab}
%      \sum\limits_{\substack{k<l\\c<d}}^{\neq ijab}
%      t_{kl}^{cd}
%      \bra{\F}
%         \hat{H}_c
%      \ket{\Phi_{kl}^{cd}}
%\\
%  t_{ij}^{ab} 
%  \sum\limits_{\substack{k<l\\c<d}}^{=ijab}
%   t_{kl}^{cd}
%   \bra{\F}
%      \hat{H}_c
%   \ket{\Phi_{kl}^{cd}} \
%&= 
%   \braket{ij || ab} 
%+
%   \sum\limits_{\substack{k<l\\c<d}}
%   t_{kl}^{cd}
%   \bra{\F}
%      \hat{H}_c
%   \ket{\Phi_{kl}^{cd}}
%\intertext{We can clearly see that each term on the lefthand side is quadratic in the t-amplitudes. In the CEPA$_0$ approximation, we ignore all of these terms under the assumption that they are small.} 
%   \cancel{
%      t_{ij}^{ab} 
%      \sum\limits_{\substack{k<l\\c<d}}^{=ijab}
%      t_{kl}^{cd}
%      \bra{\F}
%         \hat{H}_c
%      \ket{\Phi_{kl}^{cd}}
%   }
%&= 
%   \braket{ij || ab} 
%+
%   \sum\limits_{\substack{k<l\\c<d}}
%   t_{kl}^{cd}
%   \langle \Phi_{ij}^{ab} |
%      \hat{H}_c
%   \ket{\Phi_{kl}^{cd}}
%\\
%\end{align*}
%Which yields the final CEPA$_0$ amplitude equations. We can solve them iteratively to obtain the coefficients $t_{ij}^{ab}$, which we use to evaluate the CCD energy expression. This is the CEPA$_0$ energy.
%\begin{equation*}
%   \boxed{
%   0
%= 
%   \braket{ij || ab} 
%+
%   \sum\limits_{\substack{k<l\\c<d}}
%   t_{kl}^{cd}
%   \bra{\F}
%      \hat{H}_c
%   \ket{\Phi_{kl}^{cd}}
%}
%\end{equation*}

\subsection*{CEPA(1)}

For the  CEPA(1) method, we introduce the concept of pair energies:
\begin{equation*} 
   \epsilon_{ij}
=
   \frac{1}{2}
   \sum\limits_{ab}
   c_{ij}^{ab}
   \langle \Phi |
      \hat{H}_{{c}}
   | \Phi_{ij}^{ab} \rangle
\end{equation*}
Physically we can think of the pair energy as the energy contributions of those configurations that have electrons excited from the same occupied orbitals, summed over all possible excitations into virtual orbitals. \vspace{0.2cm}
Using this notation, the CEPA(1) technique approximates the unlinked contributions as follows:
\begin{equation*}
      \Delta_{{L}} 
=
   c_{ij}^{ab}
   \left(
   \epsilon_{ij} 
+
   \frac{1}{2}
   \sum\limits_{k}
      \left(
        \epsilon_{ik}
        +
        \epsilon_{kj}
      \right)
   \right)
\end{equation*}
Notably, approximation has the desired property of being exact for separated electron pairs. \vspace{0.2cm} \\
The resulting amplitude equations are
\begin{equation}
    c_{ij}^{ab}
    \left(
   \epsilon_{ij} 
+
   \frac{1}{2}
   \sum\limits_{k}
      \left(
        \epsilon_{ik}
        +
        \epsilon_{kj}
      \right)
   \right)
= 
   \langle \Phi_{ij}^{ab} |
      \hat{H}_c 
   | \Phi \rangle
+
   \langle \Phi_{ij}^{ab} |
      \hat{H}_c
      \hat{T_2}
   | \Phi \rangle
\end{equation}





\end{document}
